\begin{enumerate}[1.]
\item Os campos elétricos de duas ondas eletromagnéticas de polarização linear, propagando em unísono através do espaço livre são dados por $E_{1} = e^{-\jmath \pi (z - 0,25)} \hat{a_{x}} [\si{\volt/\meter}]$ e $E_{2} = ae^{-\jmath \pi (z - 0,25b)} \hat{a_{y}} [\si{\volt/\meter}]$ com z em metros e onde a e b são constantes. Determine para a onda eletromagnética resultante: o campo magnético instantâneo, a polarização da onda, o vetor de Poynting instantâneo e médio.  
  

    Dados: $a = 3$, $b = 1$


    Resposta:

    O campo magnético é determinado por $\vec{k} \times \vec{E} = \omega \mu_{0} \vec{H}$ $\Rightarrow$ $k\hat{a_{z}} \times (\hat{a_{x}} + 3\hat{a_{y}})e^{-\jmath (z - 0,25)\pi}$ $=$ $\dfrac{k \mu_{0}}{\sqrt{\epsilon_{0}\mu_{0}}} \vec{H}$ $\Rightarrow$ $\left[ (\hat{a_{z}} \times \hat{a_{x}}) + 3(\hat{a_{z}} \times \hat{a_{y}}) \right]e^{-\jmath (z - 0,25)\pi}$ $=$ $\sqrt{\dfrac{\mu_{0}}{\epsilon_{0}}} \vec{H}$ $\Rightarrow$ $\vec{H}$ $=$ $\sqrt{\dfrac{\epsilon_{0}}{\mu_{0}}}e^{-\jmath (z - 0,25)\pi}\left(-3\hat{a_{x}} + \hat{a_{y}}\right)$. Como $\sqrt{\dfrac{\epsilon_{0}}{\mu_{0}}}$ $=$ $\dfrac{1}{120\pi}$, temos $\vec{H}$ $=$ $\dfrac{1}{40\pi}\left(-\hat{a_{x}} + \dfrac{1}{3}\hat{a_{y}}\right)e^{-\jmath (z - 0,25)\pi} [\si{\ampere/\meter}]$. Com $k_{z} = \pi [\si{\radian/\meter}]$, temos $\omega = 3\pi \times 10^{8} [\si{\radian/\second}]$, assim $\vec{H}$ $=$ $\dfrac{1}{40\pi}\left(-\hat{a_{x}} + \dfrac{1}{3}\hat{a_{y}}\right)\cos[3\pi \times 10^{8}t - (z - 0,25)\pi] [\si{\ampere/\meter}]$

    Seja o campo elétrico $\vec{E}$ dado por $\vec{E} = E_{0}x \hat{a_{x}} + E_{0}y \hat{ a_{y}}$  $=$ $e^{-\jmath (z - 0,25) \pi} \hat{a_{x}} + 3 e^{-\jmath (z - 0,25)\pi} \hat{a_{y}} [\si{\volt/\meter}]$, reescrevendo o campo elétrico temos $\vec{E} = (\hat{a_{x}} + A e^{\jmath \theta} \hat{a_{y}})e^{-\jmath k_{z} z}$, com $A = \dfrac{E_{0}y}{E_{0}x}$ $=$ $\dfrac{3 e^{-\jmath (z - 0,25)\pi}}{e^{-\jmath (z - 0,25) \pi}} = 3$, com $\theta = \phi_{y} - \phi{x}  = 0$ e $k_{z} = \pi$, logo $\vec{E} = (\hat{a_{x}} + 3 \hat{a_{y}})e^{-\jmath \pi z}$. A polarização da onda é linear, uma reta que passa pela origem e faz um ângulo $tg^{-1}(3) \approx 71.56\si{\degree}$ com o eixo horizontal.

    O vetor de Poynting instantâneo é dado por $\vec{P}(\vec{r},t)$ $=$ $\vec{E}(\vec{r},t) \times \vec{H}(\vec{r},t)$. No problema temos $\vec{E}(\vec{r},t)$ $=$ $\left(\hat{a_{x}} + 3\hat{a_{y}}\right)\cos[3\pi \times 10^{8}t - (z - 0,25)\pi] [\si{\volt/\meter}]$ e $\vec{H}(\vec{r},t)$ $=$ $\left(-\hat{a_{x}} + \dfrac{1}{3}\hat{a_{y}}\right)\cos[3\pi \times 10^{8}t - (z - 0,25)\pi] [\si{\ampere/\meter}]$
\end{enumerate}


%%% Local Variables: 
%%% mode: latex
%%% TeX-master: "../main"
%%% End: 
