\begin{enumerate}[1.]
\item Os campos elétricos de duas ondas eletromagnéticas de polarização linear, propagando em unísono através do espaço livre são dados por $E_{1} = e^{-\jmath \pi (z - 0,25)} \hat{a_{x}} [V/m]$ e $E_{2} = ae^{-\jmath \pi (z - 0,25b)} \hat{a_{y}} [V/m]$ com z em metros e onde a e b são constantes. Determine para a onda eletromagnética resultante: o campo magnético instantâneo, a polarização da onda, o vetor de Poynting instantâneo e médio.  
  

    Dados: $a = 3$, $b = 1$


    Resposta:

    Seja o campo elétrico $\vec{E}$ dado por $\vec{E} = E_{0}x \hat{a_{x}} + E_{0}y \hat{a_{y}}$  $=$ $e^{-\jmath (z - 0,25) \pi} \hat{a_{x}} + 3 e^{-\jmath (z - 0,25)\pi} \hat{a_{y}} [V/m]$, reescrevendo o campo elétrico temos $\vec{E} = (\hat{a_{x}} + A e^{\jmath \phi} \hat{a_{y}})e^{-\jmath k_{z} z}$, com $A = \dfrac{E_{0}y}{E_{0}x}$

\end{enumerate}


%%% Local Variables: 
%%% mode: latex
%%% TeX-master: "../main"
%%% End: 
