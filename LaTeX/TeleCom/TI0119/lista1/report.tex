\begin{enumerate}[i.]
  \item Exercícios:
    \begin{enumerate}[1.]
    \item Escrever na forma $z = x + \jmath y$
      \begin{enumerate}[a)]
      \item $\dfrac{1 - \jmath}{1 + \jmath}$
      \item $\dfrac{4 - \jmath 3}{1 + \jmath}$ $-$ $\dfrac{1 - \jmath}{\sqrt{2} - \jmath}$
      \item $\left(\dfrac{1 + \jmath}{1 - \jmath}\right)^{30}$
      \end{enumerate}
    \item Mostrar que: $\|\alpha + \beta \|^{2} + \|\alpha - \beta \|^{2}$ $=$ $2\|\alpha\|^{2}$ $+$ $2\|\beta\|^{2}$
    \end{enumerate}
  \item Exercícios:
    \begin{enumerate}[1.]
    \item Determine o argumento
      \begin{enumerate}[a)]
      \item $z = - \sqrt{3} + \jmath$
      \item $z = \left(\dfrac{\jmath}{1 + \jmath \sqrt{3}}\right)^{5}$
      \item $z = - \jmath$
      \end{enumerate}
    \item Se $\|z_{1}\|$ $=$ $\|z_{2}\|$ $=$ $\|z_{3}\|$ $=$ $1$ e $z_{1} + z_{2} + z_{3} = 0$, mostrar que $z_{1}$, $z_{2}$ e $z_{3}$ são os vértices de um triângulo equilátero inscrito no círculo unitário com o centro na origem.
    \end{enumerate}
  \item Exercício:
    \begin{enumerate}[1.]
      \item Mostrar a generalização : $\displaystyle \|\sum^{N}_{k = 1} z_{k}\|$ $\leq$ $\displaystyle \sum^{N}_{k = 1} \|z_{k}\|$
    \end{enumerate}
  \item Exercícios:
    \begin{enumerate}[1.]
      \item Calcule e plote as raizes de:
        \begin{enumerate}[a)]
          \item $\sqrt{-4}$
          \item $\left(1 + \jmath \sqrt{3}\right)^{\dfrac{1}{2}}$
          \item $\sqrt[3]{- \jmath}$
          \item $\left(-1 - \jmath \sqrt{3}\right)^{\dfrac{1}{2}}$
        \end{enumerate}
      \item Resolva para $P(z) = 0$ e fatore $P(z)$
        \begin{enumerate}[a)]
          \item $P(z) = z^{6} - 64$
          \item $P(z) = z^{4} - 9$
          \item $P(z) = 5z^{3} + 8$
        \end{enumerate}
      \item Resolva $P(z) = z^{4} + \left(1 - \jmath\right) z^{2} + 2\left(1 - \jmath \right)$ para $P(z) = 0$ e plote os zeros.
    \end{enumerate}
  \item Exercícios:
    \begin{enumerate}[1.]
    \item Passe para a forma $re^{\jmath \theta}$
      \begin{enumerate}[a)]
      \item $1 + \jmath$
      \item $\left(- 1 - \jmath \sqrt{3}\right)$
      \item $(-3)$
      \end{enumerate}
    \item Mostre que $\cos{\theta} = \dfrac{e^{\jmath \theta} + e^{-\jmath \theta}}{2}$ e $\sin{\theta} = \dfrac{e^{\jmath \theta} - e^{-\jmath \theta}}{\jmath 2}$
    \item O número $\jmath^{\jmath}$ é real, imaginário ou complexo?
    \end{enumerate}
  \end{enumerate}


% \begin{pspicture}[showgrid=true](-3,-3)(3,3)
%   \psaxeslabels(0,0)(-3,-3)(3,3){$\mathfrak{Re}$}{$\mathfrak{Img}$}
%   \uput[0](0.15,0.75){Raízes de $\sqrt{-4}$}
%   \psdot(0,2) 	%plots the point (0,2)
%   \uput[0](0,2){$\jmath 2$}
%   \psdot(0,-2) 	%plots the point (0,-2)
%   \uput[0](0,-2){$-\jmath 2$}
% \end{pspicture}

\newpage

\begin{figure}[!htb]
    \centering
    \begin{minipage}{.5\textwidth}
        \centering
        \begin{pspicture}[showgrid=true](-3,-3)(3,3)
          \psaxeslabels(0,0)(-3,-3)(3,3){$\mathfrak{Re}$}{$\mathfrak{Img}$}
          \psdot(0,2) 	%plots the point (0,2)
          \uput[0](0,2){$\jmath 2$}
          \psdot(0,-2) 	%plots the point (0,-2)
          \uput[0](0,-2){$-\jmath 2$}
        \end{pspicture}
        \caption{Raízes de $\sqrt{-4}$}
        \label{fig:fig_1}
    \end{minipage}%
    \begin{minipage}{0.5\textwidth}
        \centering
        \begin{pspicture}[showgrid=true](-3,-3)(3,3)
          \psaxeslabels(0,0)(-3,-3)(3,3){$\mathfrak{Re}$}{$\mathfrak{Img}$}
          \psdot(1.2247,0.7071) 	%plots the point (0,2)
          \uput[0](0.1,1.5){$\dfrac{\sqrt{2}}{2} (\sqrt{3} + \jmath)$}
          \psdot(-1.2247,-0.7071) 	%plots the point (0,-2)
          \uput[0](-3,-1.5){$-\dfrac{\sqrt{2}}{2} (\sqrt{3} + \jmath)$}
        \end{pspicture}
        \caption{Raízes de $(1 + \jmath \sqrt{3})^{\dfrac{1}{2}}$}
        \label{fig:fig_2}
    \end{minipage}
\end{figure}

\begin{figure}[!htb]
    \centering
    \begin{minipage}{.5\textwidth}
        \centering
        \begin{pspicture}[showgrid=true](-3,-3)(3,3)
          \psaxeslabels(0,0)(-3,-3)(3,3){$\mathfrak{Re}$}{$\mathfrak{Img}$}
          \psdot(0.8660,0.5) 	%plots the point (0,2)
          \uput[0](1.1,1){$\dfrac{1}{2}(\sqrt{3} - \jmath)$}
          \psdot(0,1) 	%plots the point (0,-2)
          \uput[0](0,1){$\jmath$}
          \psdot(-0.8660,-0.5) 	%plots the point (0,2)
          \uput[0](-2.4,-1){$-\dfrac{1}{2}(\sqrt{3} - \jmath)$}
        \end{pspicture}
        \caption{Raízes de $\sqrt[3]{-\jmath}$}
        \label{fig:fig_3}
    \end{minipage}%
    \begin{minipage}{0.5\textwidth}
        \centering
        \begin{pspicture}[showgrid=true](-3,-3)(3,3)
          \psaxeslabels(0,0)(-3,-3)(3,3){$\mathfrak{Re}$}{$\mathfrak{Img}$}
          \psdot(0.7071,-1.2247) 	%plots the point (0.8660,0.5)
          \uput[0](0.3,-1.75){$\dfrac{\sqrt{2}}{2}(1 - \jmath \sqrt{3})$}
          \psdot(-0.7071,1.2247) 	%plots the point (0,-1)
          \uput[0](-2.9,0.75){$\dfrac{\sqrt{2}}{2}(-1 + \jmath \sqrt{3})$}
        \end{pspicture}
        \caption{Raízes de $(-1 - \jmath \sqrt{3})$}
        \label{fig:fig_4}
    \end{minipage}
\end{figure}

\begin{figure}[!htb]
    \centering
    \begin{pspicture}[showgrid=true](-3,-3)(3,3)
      \psaxeslabels(0,0)(-3,-3)(3,3){$\mathfrak{Re}$}{$\mathfrak{Img}$}
      \psdot(1,1) 	%plots the point (0,2)
      \uput[0](1,0.5){$1 + \jmath$}
      \psdot(-1,-1) 	%plots the point (0,-2)
      \uput[0](-1.6,0.5){$-1 - \jmath$}
      \psdot(0.4550,-1.0986) 	%plots the point (0,2)
      \uput[0](0.6,-1){$\sqrt[4]{2} e^{-\jmath 67.5^{o}}$}
      \psdot(-0.4550,1.0986) 	%plots the point (0,2)
      \uput[0](-2.4,-1.3){$\sqrt[4]{2} e^{-\jmath 112.5^{o}}$}
    \end{pspicture}
    \caption{Raízes de $P(Z) = Z^4 + (1 -\jmath)Z^2 + 2(1 -\jmath)$}
    \label{fig:fig_3}
\end{figure}