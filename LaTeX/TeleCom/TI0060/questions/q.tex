\section*{O que é uma Banda?}
\label{q2}

É a faixa permitida de energia, na qual elétrons podem ocupar orbitais
em um átomo.

\section*{Por que surgem bandas proibidas em um cristal?}
\label{q3}

Os  elétrons,  em   um  cristal,  vibram  de   forma  periódica.  Essa
característica faz com que as soluções da equação de onda $\Psi$ sejam
descontínuas em  determinados intervalos, ditas bandas  proibidas, que
são a diferença de energia entre duas bandas permitidas.

\section*{Como é feita a formação do estado fundamental de um sólido?}
\label{q4}

Pelo preenchimento, dos elétrons, nos níveis mais baixos de energia em
cada orbital.  A forma como isso  ocorre é que determina  o sólido ser
condutor ou isolante elétrico.

\section*{O que é a banda de valência?}
\label{q5}

É banda que possui o maior nível energético no átomo, ou seja é a
última a ser preenchida quando o átomo está no modo fundamental.

\section*{O que é a banda de condução?}
\label{q6}

É a banda que permite que um elétron se desloque da camada de valência de um
átomo para o próximo átomo formando uma corrente de elétrons.

\section*{O que é um semicondutor?}
\label{q7}

É um sólido  que possui um menor índice de  condutividade elétrica, em
relação aos  bons condutores,  e maior  condutividade, em  relação aos
isolantes. Ou seja, sua banda proibida,  entre a banda de valência e a
de  condução,  possui  um  valor  energético  maior  do  que  os  bons
condutores e menor que os isolantes.

\section*{Qual o significado da distribuição de Fermi-Dirac? Qual o
  formato da distribuição em $T = 0$ K?}
\label{q8}

Descreve a probabilidade de preenchimento de um determinado estado
quântico de Energia $E$ a uma temperatura $T$. A função degrau.

\section*{Quais os fenômenos físicos que causam a resistividade?}
\label{q9}

A temperatura, que aumenta a agitação dos átomos do material, aumentando a colisão
do fluxo de elétrons com os estrutura cristalina. E o tempo de colisão
que é inversamente proporcional com a temperatura.

\section*{O que é corrente de deriva (\emph{drift})?}
\label{q10}

É o deslocamento ordenado de elétrons livres, na estrutura cristalina
de um condutor, na direção de um campo elétrico, externo, aplicado
sobre o condutor.

\section*{O que significa o $\tau$ na equação $\sigma = \dfrac{Ne^{2}\tau}{m^{*}}$?
  Calcule $\tau$ para o \emph{Cu}.}
\label{q11}

O tempo médio para o fim do deslocamento da esfera de fermi.

\vspace*{0.5cm}

Para o \emph{Cu} temos:

\begin{eqnarray*}
  \sigma & = & 5,96 \times 10^{4} [\si{\siemens/\meter^{3}}]\\ \nonumber
  N & = & \dfrac{13,6 \times 10^{9} [\si{\coulomb/\meter^{3}}]}
          {1,602 \times 10^{-19}[\si{\coulomb}]} \\ \nonumber
  & = & 8,48 \times 10^{28} \mathrm{eletrons}/\si{\meter^{3}} \\ \nonumber
  m^{*} & = & m_{e} \times 1.01 \\ \nonumber
  & = & 9,191 \times 10^{-31} \nonumber
\end{eqnarray*}

Assim o valor de $\tau = \dfrac{\sigma m^{*}}{N e^{2}}$ $\Rightarrow$ $\tau = \dfrac{9.191
  \times 10^{-31} \times 5,96 \times 10^{4}}{8,48 \times 10^{28} \times (-1.602 \times 10^{-19})^{2}}$,
assim $\tau \approx 2,51 \times 10^{-17} [\si{\second}]$

\section*{Seja um fio de prata em $T = 300 \si{\kelvin}$, com $\tau = 3.8
  \times 10^{-14} \si{\second}$ e $N = 5,86 \times 10^{28} \si{\meter^{-3}}$. $R
  = \sigma \dfrac{L}{A}$. $V_{deriva} = E \dfrac{e \tau}{m^{*}}$, sendo E o campo
  elétrico. Calcule:}
\label{q12}

\begin{enumerate}[a)]
  \item R se $L = 100 \si{\meter}$ e $A = 10^{-7} \si{\meter^{2}}$.
    \begin{eqnarray*}
      R & = & \sigma \dfrac{L}{A} \\ \nonumber
        & = & 6,3 \times 10^{4} \dfrac{100}{10^{-7}} \\ \nonumber
        & = & 6,3 \times 10^{13} \si{\ohm}\nonumber
    \end{eqnarray*}
  \item A corrente $I$ se a tensão aplicada nas extremidades do fio é
    de $1,6 \si{\volt}$.
    \begin{eqnarray*}
      I & = & \dfrac{V}{R} \\ \nonumber
        & = & \dfrac{1,6}{6,3 \times 10^{13}} \\ \nonumber
        & = & 25,3 \times 10^{-15} \si{\ampere}\nonumber
    \end{eqnarray*}
  \item A velocidade de deriva do elétron livre.
    \begin{eqnarray*}
      V_{deriva} & = &  \dfrac{e \tau}{m^{*}} \\ \nonumber
               & = & 1,6 \times 10^{-2} \dfrac
                       {(-1.602 \times 10^{-19}) \times 3.8 \times 10^{-14}}
                       {0.99 \times 9,1 \times 10^{-31}} \\ \nonumber
      V_{deriva} & = & - 1,08 \times 10^{-4} [\si{\meter/\second}] \nonumber
    \end{eqnarray*}
  \end{enumerate}

\section*{O que  é um semicondutor  de gap  direto e indireto?  Qual é
melhor apropriado para o uso em optoeletrônica e por que?}
\label{q13}

A diferença entre os dois tipos vem do fato que no semicondutor de gap
direto o estado de  mínima energia da banda de condução  e o de máxima
energia da camada de valência ocorrem exatamente para o mesmo valor do
momento,  emissão  ou absorvição  de  fótons.   No indireto  isso  não
acontece, pois além dos fótons exitem a presença de fônons, emissão ou
absorvição, aumentando a duração do processo.


\noindent O de gap direto é melhor apropriado para o uso, dado que para ocorrer
a transição da banda de valência  para a banda de condução, da maneira
mais eficiente  possível.

\section*{O que é um buraco?}
\label{q14}

É o estado de energia na banda de valência que fica não preenchido,
dado que o elétron, que ocupava este estado, fez a transição para a
banda de condução.

\section*{Por que em um condutor intrínseco o número de buracos é
  igual ao número de elétrons?}
\label{q15}

Em um semicondutor intrínseco a banda de valência é completamente
preenchida, quando um elétron faz a transição para a banda de condução
automaticamente o estado energético, ocupado por este elétron, fica
desocupado, ou seja, surge um buraco na banda de valência.

\section*{Calcule $f(E)$ para um estado com energia $E = E_{f} + 0,2
  \, \si{\electronvolt}$, em $T = 290 \si{\kelvin}$, usando as equações:}
\label{q16}

\begin{enumerate}[i)]
\item $e^{\dfrac{-(E - E_{f})}{K_{B}T}}$
  \begin{eqnarray*}
    f(E_{f} + 0,2) & = & e^{{\dfrac{(-0,2)}
                         {1,38 \times 10^{-23} \times 290}}} \\
    \nonumber
                  & = & e^{-5 \times 10^{19}} \\ \nonumber
    f(E_{f} + 0,2) & \approx & 0
  \end{eqnarray*}
\item $\dfrac{1}{1 + e^{\dfrac{(E - E_{f})}{K_{B}T}}}$
  \begin{eqnarray*}
    f(E_{f} + 0,2) & = & \dfrac{1}{1 + e^{\frac{0,2}
                         {1,38 \times 10^{-23} \times 290}}} \\
    \nonumber
    f(E_{f} + 0,2) & = & \dfrac{1}{2} \nonumber
  \end{eqnarray*}
\end{enumerate}

\section*{}
\label{q17}