\section*{Questão 1}
\label{q1}

\begin{enumerate}[i.]
   \item O  gráfico dos  níveis e  energia de cada  banda presente  em um
átomo de um  sólido cristalino. As áreas hachuradas  mostram os níveis
de energia permitidos, aonde os elétrons deveram preenche-los conforme
seu níveis energéticos.
   \item Mostra a  estrutura de bandas de um isolante.  Onde as bandas de
valência e condução estão  totalmente preenchidas ou vazias. Impedindo
a movimentação entre as bandas
   \item 
   \end{enumerate}
   
\section*{O que é uma Banda?}
\label{q2}

É a faixa permitida de energia, na qual elétrons podem ocupar orbitais
em um átomo.

\section*{Por que surgem bandas proibidas em um cristal?}
\label{q3}

Os  elétrons,  em   um  cristal,  vibram  de   forma  periódica.  Essa
característica faz com que as soluções da equação de onda $\Psi$ sejam
descontínuas em  determinados intervalos, ditas bandas  proibidas, que
são a diferença de energia entre duas bandas permitidas.

\section*{Como é feita a formação do estado fundamental de um sólido?}
\label{q4}

Pelo preenchimento, dos elétrons, nos níveis mais baixos de energia em
cada orbital.  A forma como isso  ocorre é que determina  o sólido ser
condutor ou isolante elétrico.

\section*{O que é a banda de valência?}
\label{q5}

É banda que possui o maior nível energético no átomo, ou seja é a
última a ser preenchida quando o átomo está no modo fundamental.

\section*{O que é a banda de condução?}
\label{q6}

É a banda que permite que um elétron se desloque da camada de valência de um
átomo para o próximo átomo formando uma corrente de elétrons.

\section*{O que é um semicondutor?}
\label{q7}

É um sólido  que possui um menor índice de  condutividade elétrica, em
relação aos  bons condutores,  e maior  condutividade, em  relação aos
isolantes. Ou seja, sua banda proibida,  entre a banda de valência e a
de  condução,  possui  um  valor  energético  maior  do  que  os  bons
condutores e menor que os isolantes.

\section*{Qual o significado da distribuição de Fermi-Dirac? Qual o
  formato da distribuição em $T = 0$ K?}
\label{q8}

Descreve a probabilidade de preenchimento de um determinado estado
quântico de Energia $E$ a uma temperatura $T$. A função degrau.

\section*{Quais os fenômenos físicos que causam a resistividade?}
\label{q9}

A temperatura, que aumenta a agitação dos átomos do material, aumentando a colisão
do fluxo de elétrons com os estrutura cristalina. E o tempo de colisão
que é inversamente proporcional com a temperatura.

\section*{O que é corrente de deriva (\emph{drift})?}
\label{q10}

É o deslocamento ordenado de elétrons livres, na estrutura cristalina
de um condutor, na direção de um campo elétrico, externo, aplicado
sobre o condutor.

\section*{O que significa o $\tau$ na equação $\sigma = \dfrac{Ne^{2}\tau}{m^{*}}$?
  Calcule $\tau$ para o \emph{Cu}.}
\label{q11}

O tempo médio para o fim do deslocamento da esfera de fermi.

\vspace*{0.5cm}

Para o \emph{Cu} temos:

\begin{eqnarray*}
  \sigma & = & 5,96 \times 10^{4} [\si{\siemens/\meter^{3}}]\\ \nonumber
  N & = & \dfrac{13,6 \times 10^{9} [\si{\coulomb/\meter^{3}}]}
          {1,602 \times 10^{-19}[\si{\coulomb}]} \\ \nonumber
  & = & 8,48 \times 10^{28} \mathrm{eletrons}/\si{\meter^{3}} \\ \nonumber
  m^{*} & = & m_{e} \times 1.01 \\ \nonumber
  & = & 9,191 \times 10^{-31} \nonumber
\end{eqnarray*}

Assim o valor de $\tau = \dfrac{\sigma m^{*}}{N e^{2}}$ $\Rightarrow$ $\tau = \dfrac{9.191
  \times 10^{-31} \times 5,96 \times 10^{4}}{8,48 \times 10^{28} \times (-1.602 \times 10^{-19})^{2}}$,
assim $\tau \approx 2,51 \times 10^{-17} [\si{\second}]$

\section*{Seja um fio de prata em $T = 300 \si{\kelvin}$, com $\tau = 3.8
  \times 10^{-14} \si{\second}$ e $N = 5,86 \times 10^{28} \si{\meter^{-3}}$. $R
  = \sigma^{-1} \dfrac{L}{A}$. $V_{deriva} = E \dfrac{e \tau}{m^{*}}$, sendo E o campo
  elétrico. Calcule:}
\label{q12}

\begin{enumerate}[a)]
  \item R se $L = 100 \si{\meter}$ e $A = 10^{-7} \si{\meter^{2}}$.
    \begin{eqnarray*}
      R & = & \sigma^{-1} \dfrac{L}{A} \\ \nonumber
        & = & \dfrac{100}{6,3 \times 10^{6} \times 10^{-7}} \\ \nonumber
        & = & 158,73 \si{\ohm}\nonumber
    \end{eqnarray*}
  \item A corrente $I$ se a tensão aplicada nas extremidades do fio é
    de $1,6 \si{\volt}$.
    \begin{eqnarray*}
      I & = & \dfrac{V}{R} \\ \nonumber
        & = & \dfrac{1,6}{158,73} \\ \nonumber
        & = & 10,08 \si{\milli\ampere}\nonumber
    \end{eqnarray*}
  \item A velocidade de deriva do elétron livre.
    \begin{eqnarray*}
      V_{deriva} & = &  \dfrac{e \tau}{m^{*}} \\ \nonumber
               & = & 1,6 \times 10^{-2} \dfrac
                       {(-1.602 \times 10^{-19}) \times 3.8 \times 10^{-14}}
                       {0.99 \times 9,1 \times 10^{-31}} \\ \nonumber
      V_{deriva} & = & - 1,08 \times 10^{-4} [\si{\meter/\second}] \nonumber
    \end{eqnarray*}
  \end{enumerate}

\section*{O que  é um semicondutor  de gap  direto e indireto?  Qual é
melhor apropriado para o uso em optoeletrônica e por que?}
\label{q13}

A diferença entre os dois tipos vem do fato que no semicondutor de gap
direto o estado de  mínima energia da banda de condução  e o de máxima
energia da camada de valência ocorrem exatamente para o mesmo valor do
momento,  emissão  ou absorvição  de  fótons.   No indireto  isso  não
acontece, pois além dos fótons exitem a presença de fônons, emissão ou
absorvição, aumentando a duração do processo.


\noindent O de gap direto é melhor apropriado para o uso, dado que para ocorrer
a transição da banda de valência  para a banda de condução, da maneira
mais eficiente  possível.

\section*{O que é um buraco?}
\label{q14}

É o estado de energia na banda de valência que fica não preenchido,
dado que o elétron, que ocupava este estado, fez a transição para a
banda de condução.

\section*{Por que em um condutor intrínseco o número de buracos é
  igual ao número de elétrons?}
\label{q15}

Em um semicondutor intrínseco a banda de valência é completamente
preenchida, quando um elétron faz a transição para a banda de condução
automaticamente o estado energético, ocupado por este elétron, fica
desocupado, ou seja, surge um buraco na banda de valência.

\section*{Calcule $f(E)$ para um estado com energia $E = E_{f} + 0,2
  \, \si{\electronvolt}$, em $T = 290 \si{\kelvin}$, usando as equações:}
\label{q16}

\begin{enumerate}[i)]
\item $e^{\dfrac{-(E - E_{f})}{K_{B}T}}$
  \begin{eqnarray*}
    f(E_{f} + 0,2) & = & e^{{\dfrac{(-0,2)}
                         {1,38 \times 10^{-23} \times 290}}} \\
    \nonumber
                  & = & e^{-5 \times 10^{19}} \\ \nonumber
    f(E_{f} + 0,2) & \approx & 0
  \end{eqnarray*}
\item $\dfrac{1}{1 + e^{\dfrac{(E - E_{f})}{K_{B}T}}}$
  \begin{eqnarray*}
    f(E_{f} + 0,2) & = & \dfrac{1}{1 + e^{\frac{0,2}
                         {1,38 \times 10^{-23} \times 290}}} \\
    \nonumber
    f(E_{f} + 0,2) & \approx & 0 \nonumber
  \end{eqnarray*}
\end{enumerate}

\section*{Explique tudo o que você entende sobre semicondutor dopado com impurezas
doadoras (o que acontece quimicamente, quem é o portador majoritário, como fica o
nível de Fermi, etc...)}
\label{q17}

O semicondutor tem suas propriedades elétricas modificadas pela adição
de impurezas,  processo de dopagem,  para se obter  uma característica
desejada. Quimicamente  a impureza  adiciona elétrons,  excedentes, na
banda de  valência do  semicondutor. Sendo estes  elétrons excedentes,
provenientes da impureza, chamados de portador majoritário.

\vspace*{0.5cm}

\noindent Dado ao excesso de elétrons e falta de buracos o nível de \emph{Fermi}
está mais próximo da banda de condução, assim $\dfrac{(E_{f} - E_{i})}{K_{B}T} >> 1$.

\section*{Explique tudo o que você entende sobre semicondutor dopado com impurezas
aceitadoras (o que acontece quimicamente, quem é o portador majoritário, como fica o
nível de Fermi, etc...)}
\label{q18}

O semicondutor tem suas propriedades elétricas modificadas pela adição
de impurezas,  processo de dopagem,  para se obter  uma característica
desejada.   Quimicamente  a  impureza  retira  elétrons  da  banda  de
valência do  semicondutor aumentando a ocorrência  de buracos, estados
energéticos  não  preenchidos.  Logo  os  buracos  são  os  portadores
majoritários.

\vspace*{0.5cm}

\noindent Dado ao excesso de buracos e ausência de elétrons o nível de
\emph{Fermi}   está  mais   próximo  da   banda  de   valência,  assim
$\left|\left|\dfrac{(E_{f} - E_{i})}{K_{B}T}\right|\right| >> 1$.

\section*{Questão 19}
\label{q19}


Seja  um semicondutor  tipo n  com  $N_{d}$ impurezas  doadoras a  uma
temperatura tal  que todas estejam  ionizadas, ou seja  $N_{d} \approx
N^{+}_{d}$. Neste caso,  a neutralidade total da carga  nos afirma que
$n_{0} ~ p_{0} + N_{d}$.  Usando a lei de ação das massas ($n_{0}p_{0}
= n^{2}_{i}$), temos

$$
n_{0} =  \dfrac{N_{d}}{2} +  \left[\left(\dfrac{N_{d}}{2}\right)^{2} +
  n_{i}^{2}\right]^{\frac{1}{2}}
p_{0} = - \dfrac{N_{d}}{2} + \left[\left(\dfrac{N_{d}}{2}\right)^{2} +
n_{i}^{2}\right]^{\frac{1}{2}}
$$

Normalmente,  no semicondutor  dopado  $N_{d} >>  n_{i}$, neste  caso,
fazendo simplificações  obtemos: $$n_{0} ~ N_{d}  \,\, \mathrm{e} \,\,
p_{0} \approx \dfrac{n^{2}_{i}}{N_{d}}.$$

Para determinar o nível de \emph{Fermi}, fazemos:

\begin{eqnarray*}
  n_{0} & = & N_{c} e^{\dfrac{-(E_{c} - E_{f})}{K_{B}T}} \\ \nonumber
       & = & N_{d} \\ \nonumber
  E_{f} & = & E_{c} - K_{B}T \ln{\dfrac{N_{c}}{N_{v}}} \nonumber
\end{eqnarray*}

\begin{eqnarray*}
  n_{0} & = & n_{i} e^{\dfrac{(E_{f} - E_{i})}{K_{B}T}} \\ \nonumber
       & = & N_{d} \\ \nonumber
  E_{f} & = & E_{i} + K_{B}T \ln{\dfrac{N_{d}}{N_{i}}} \nonumber
\end{eqnarray*}

Mostre passo a  passo que, para um semicondutor  tipo \emph{p}, dopado
com impurezas  aceitadoras, as  expressões para  as concentrações  e o
nível de Fermi são:

\begin{eqnarray*}
  n_{0} & \approx & \dfrac{n^{2}_{i}}{N_{a}} \\ \nonumber
  p_{0} & \approx & N_{a} \\ \nonumber
  E_{f} & = & E_{v} + K_{B}T \ln{\dfrac{N_{v}}{N_{a}}} \\ \nonumber
  E_{f} & = & E_{i} - K_{B}T \ln{\dfrac{N_{a}}{n_{i}}} \nonumber
\end{eqnarray*}

usando:

\begin{eqnarray*}
  p_{0} & = & N_{v} e^{\dfrac{-(E_{f} - E_{v})}{K_{B}T}} \\ \nonumber
  p_{0} & = & n_{i} e^{\dfrac{(E_{i} - E_{f})}{K_{B}T}} \nonumber
\end{eqnarray*}

\section*{Quem   apresenta   maior   condutividade,   o   semicondutor
intrínseco ou o dopado? Por que?}
\label{q20}

O  extrínseco,  dopado,  possui   maior  condutividade.  Dado  que  as
impurezas  adicionadas ao  material  acrescentam  elétrons ou  buracos
excedentes, no  material, permitem que ocorram  uma densidade superior
de portadores  de carga  na banda de  valência. Assim,  permitindo uma
maior mobilidade  de elétrons entre a  banda de valência e  a banda de
condução.

\section*{Calcule as concentrações  de elétrons e buracos  e a posição
do nível  de Fermi  (em relação  à energia da  banda de  condução) num
cristal de silício dopado com $10^{16} \si{\centi\meter^{-3}}$ átomos de
$A_{s}$,  à  temperatura  $T  = 290  \si{\kelvin}$  ($K_{B}T  =  0,025
\si{\electronvolt}$).}
\label{q21}

