\begin{enumerate}[1.]
% 1a. Questão  
\item Uma linha de transmissão sem perdas com C $= 7 \times 10^{-11}$[\si{\farad/\meter}] e L $= 5 \times 10^{-7}$[\si{\henry/\meter}], tem $40$\si{\meter} de comprimento e uma carga Z$_{L} = 40$\si{\ohm}. Se uma fonte ideal de tensão fornece $100$\si{\volt} na entrada da linha e opera numa frequência de $10$\si{\mega\hertz} a $12$\si{\mega\hertz}, determinar as curvas da corrente de entrada da linha e a corrente na carga em função da frequência (intensidade e fase).
% 2a. Questão   
\item Uma carga Z$_{L} = (80 + \jmath 100)$\si{\ohm} está conectada a uma linha de transmissão sem perdas com Z$_{0} = 50$\si{\ohm}, usando a carta de Smith determine:

\begin{enumerate}[a.]
  \setlength\itemindent{15pt} \item $\Gamma$
  \setlength\itemindent{15pt} \item TOE
  \setlength\itemindent{15pt} \item A admitância da carga Y$_{L}$
  \setlength\itemindent{15pt} \item A impedância a $0,15 \lambda$ da carga
  \setlength\itemindent{15pt} \item A localização de V$_{max}$ e V$_{min}$ em relação a carga, se a linha tiver um comprimento de $0,35 \lambda$
  \setlength\itemindent{15pt} \item A impedância de entrada da linha
\end{enumerate}
% 3a. Questão
\item Uma linha de transmissão sem perdas com Z$_{0} = 50$\si{\ohm} tem $14$\si{\meter} de comprimento e opera em $16$\si{\mega\hertz}. A velocidade de propagação na linha é de $2,8 \times 10^{8}$\si{\meter/\second}. Se a linha está terminada por uma carga Z$_{L} = (175 + \jmath 100)$\si{\ohm}, use as expressões analíticas para obter:
\begin{enumerate}[a.]
  \setlength\itemindent{15pt} \item As posições do $1^{\b{o}}$ máximo e do $1^{\b{o}}$ mínimo
  \setlength\itemindent{15pt} \item A impedância de entrada da linha
\end{enumerate}
Comprovar usando a carta de Smith.

% 4a. Questão
\item Uma rede de casamento, utilizando um elemento reativo em série com um comprimento $d$ de uma LT, é utilizada para casar uma carga Z$_{L} = (80 + \jmath 100)$\si{\ohm} em uma LT com impedância de Z$_{0} = 50$\si{\ohm} operando a $18$\si{\giga\hertz}. Determinar o comprimento de linha $d$ e o valor do elemento reativo se:
\begin{enumerate}[a.]
  \setlength\itemindent{15pt} \item Um capacitor em série for utilizado
  \setlength\itemindent{15pt} \item Um indutor em série for utilizado
\end{enumerate}
% 5a. Questão
\item Projete duas redes de casamento uma por toco paralelo em aberto e outra por toco paralelo em curto para casar uma carga Z$_{L} = (80 + \jmath 100)$\si{\ohm} com uma LT com impedância de Z$_{0} = 50$\si{\ohm}. Supondo agora que a carga mudou para Z$_{L} = (85 - \jmath 100)$\si{\ohm}, determinar o coeficiente de reflexão visto na rede de casamento. Entregar as cartas de Smith utilizadas.
\end{enumerate}


% LocalWords:  Smith pt max LT
