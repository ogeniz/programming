\documentclass[a4paper,12pt]{article}
\usepackage[T1]{fontenc}
\usepackage[utf8]{inputenc}
\usepackage[brazil]{babel}
\usepackage{amsmath,amsfonts,amssymb,enumerate,MnSymbol}
\usepackage{pst-sigsys,pst-all}
\usepackage[dvips]{graphicx}
\usepackage{float,caption,subcaption}
%\usepackage{indentfirst,siunitx,multicol}
%\usepackage{booktabs,pstricks-add,pst-coil,siunitx,multirow,steinmetz}
%\usepackage[usenames,dvipsnames,svgnames,table,x11names]{xcolor}
%\usepackage[brazil]{varioref}
%\usepackage{pstricks-add,subfig,caption,url,verbatim,textcomp}
%\usepackage[cdot,squaren,Gray]{SIunits}
%\usepackage[framed, numbered, autolinebreaks, useliterate]{mcode}
% Setup pages
\setlength{\hoffset}{-3cm}
\setlength{\voffset}{-3cm}
\setlength{\textheight}{25.0cm}
\setlength{\textwidth}{19.5cm}
\pagestyle{empty}
% Setup Table Cells
% \newcolumntype{L}[1]{>{\raggedright\let\newline\\\arraybackslash\hspace{0pt}}m{#1}}
% \newcolumntype{C}[1]{>{\centering\let\newline\\\arraybackslash\hspace{0pt}}m{#1}}
% \newcolumntype{R}[1]{>{\raggedleft\let\newline\\\arraybackslash\hspace{0pt}}m{#1}}
%
\hyphenation{}
%
\newtheorem{defin}{Definition}[section]
\newtheorem{teorema}{Theorem}[section]
\renewcommand{\labelitemi}{$\bullet$}
\renewcommand{\labelitemii}{$\cdot$}
\renewcommand{\labelitemiii}{$\diamond$}
\renewcommand{\labelitemiv}{$\ast$}
%
%\setlist[itemize,1]{leftmargin=\dimexpr 26pt-.5in}
%
%\graphicspath{{figures/}}
%\DeclareGraphicsExtensions{.eps}
%%\listfiles
%
\begin{document}
{
%
     {\sf
       \vspace*{4cm}
      \begin{center}
%         {\Large{\bfseries Universidade Federal do Ceará}}\\
%         \vspace*{1.0cm}
%         {\Large{\bfseries Centro de Tecnologia}}\\
%         \vspace*{1.0cm}
         {\Large {\bfseries Departamento de Engenharia de Engenharia de Teleinformática}}\\
         \vspace*{1.0cm}
         {\Large{\bfseries Disciplina:TH$0119$ -- PDS}}\\
         \vspace*{1.0cm}
%         %%{\large{\bfseries Introdução à Instrumentação e aos Princípios Básicos da Eletricidade}}
          %%{\large{\bfseries Ondas planas em meios com e sem perdas}}
<<<<<<< HEAD
         {\large{\bfseries 2a. Lista de Exercícios}}
=======
         {\large{\bfseries TC - Convolução Circular}}
>>>>>>> 5961b38f55ae8b9c7785c30e4a57928189a5e71a
     \end{center}
% %
 \vspace*{12.0cm}
    {\Large
        \begin{flushleft}
	       \noindent Aluno: Ogeniz Façanha Costa\\
	       Curso: Engenharia de Telecomunicações\\
	       Matrícula: 371987\\
%	       Disciplina:TH$0119$ -- PDS
        \end{flushleft}}
    }
%
\newpage
%
% \addcontentsline{toc}{section}{Introdução Teórica}
% \addcontentsline{toc}{section}{Prática}
% \addcontentsline{toc}{section}{Conclusão}
% \tableofcontents
% \listoftables
<<<<<<< HEAD
% \begin{center}
	\fbox{\large{Convolução}}
\end{center}


A definição matemática de convolução vem da análise funcional, sendo este um operador linear dadas duas funções quaisquer resulta-se em uma terceira que é a medida da área superposta, a soma, entre as mesmas em função do deslocamento entre estas. \noindent Para sequências discretas $x[n]$ e $h[n]$ com $n \in \mathbb{Z}$ a convolução é definida por: $$\displaystyle y[n] = x[n] * h[n] = \sum_{k = -\infty}^{k = \infty} x[k] \cdot h[n -k]$$

No estudo de sistemas lineares e invariantes no tempo a convolução descreve a saída de um dado sistema sendo a sinal de entrada convolvido com a resposta ao impulso do sistema. Seja um sistema causal com sua resposta ao impulso dada por $h[n] = 2\delta[n] + \delta[n-1]$ e entrada $x[n] = 3\delta[n] + 4\delta[n-1] + 5\delta[n-2]$

\begin{figure*}[hb]
  \centering
  \begin{subfigure}[h]{0.3\textwidth}
    \begin{center}
      \begin{pspicture}(0,0)(3,5.75)
        \psaxes{-}(0,0)(3,0)
        \psline[linewidth=1pt,showpoints=true](0,0)(0,3) % x[n]
        \psline[linewidth=1pt,showpoints=true](1,0)(1,4) % x[n-1]
        \psline[linewidth=1pt,showpoints=true](2,0)(2,5) % x[n-2]
        \rput(0,3.5){3} % weight of x[n]
        \rput(1,4.5){4} % weight of x[n-1]
        \rput(2,5.5){5} % weight of x[n-2]
      \end{pspicture}
    \end{center}
    \caption{x[n]}
  \end{subfigure}
  %
  \begin{subfigure}[h]{0.3\textwidth}
    \begin{center}
      \begin{pspicture}(0,0)(3,5.75)
        \psaxes{-}(0,0)(3,0)
        \psline[linewidth=1pt,showpoints=true](0,0)(0,2) % h[n]
        \psline[linewidth=1pt,showpoints=true](1,0)(1,1) % h[n-1]
        \rput(0,2.5){2} % weight of h[n]
        \rput(1,1.5){1} % weight of h[n-1]
      \end{pspicture}
    \end{center}
    \caption{h[n]}
  \end{subfigure}
  \caption{Sinais de entrada x[n] e resposta ao impulso do sistema h[n]}
\end{figure*}

%%% Local Variables:
%%% mode: latex
%%% TeX-master: "../main"
%%% End:

\begin{center}
	\fbox{\large{Convolução}}
\end{center}


A definição matemática de convolução vem da análise funcional, sendo este um operador linear dadas duas funções quaisquer resulta-se em uma terceira que é a medida da área superposta, a soma, entre as mesmas em função do deslocamento entre estas. \noindent Para sequências discretas $x[n]$ e $h[n]$ com $n \in \mathbb{Z}$ a convolução é definida por: $$\displaystyle y[n] = x[n] * h[n] = \sum_{k = -\infty}^{k = \infty} x[k] \cdot h[n -k]$$

No estudo de sistemas lineares e invariantes no tempo a convolução descreve a saída de um dado sistema sendo a sinal de entrada convolvido com a resposta ao impulso do sistema. Seja um sistema causal com sua resposta ao impulso dada por $h[n] = 2\delta[n] + \delta[n-1]$ e entrada $x[n] = 3\delta[n] + 4\delta[n-1] + 5\delta[n-2]$

\begin{figure*}[hb]
  \centering
  \begin{subfigure}[h]{0.3\textwidth}
    \begin{center}
      \begin{pspicture}(0,0)(3,5.75)
        \psaxes{-}(0,0)(3,0)
        \psline[linewidth=1pt,showpoints=true](0,0)(0,3) % x[n]
        \psline[linewidth=1pt,showpoints=true](1,0)(1,4) % x[n-1]
        \psline[linewidth=1pt,showpoints=true](2,0)(2,5) % x[n-2]
        \rput(0,3.5){3} % weight of x[n]
        \rput(1,4.5){4} % weight of x[n-1]
        \rput(2,5.5){5} % weight of x[n-2]
      \end{pspicture}
    \end{center}
    \caption{x[n]}
  \end{subfigure}
  %
  \begin{subfigure}[h]{0.3\textwidth}
    \begin{center}
      \begin{pspicture}(0,0)(3,5.75)
        \psaxes{-}(0,0)(3,0)
        \psline[linewidth=1pt,showpoints=true](0,0)(0,2) % h[n]
        \psline[linewidth=1pt,showpoints=true](1,0)(1,1) % h[n-1]
        \rput(0,2.5){2} % weight of h[n]
        \rput(1,1.5){1} % weight of h[n-1]
      \end{pspicture}
    \end{center}
    \caption{h[n]}
  \end{subfigure}
  \caption{Sinais de entrada x[n] e resposta ao impulso do sistema h[n]}
\end{figure*}

%%% Local Variables:
%%% mode: latex
%%% TeX-master: "../main"
%%% End:

=======
\begin{center}
	\fbox{\large{Convolução}}
\end{center}


A definição matemática de convolução vem da análise funcional, sendo este um operador linear dadas duas funções quaisquer resulta-se em uma terceira que é a medida da área superposta, a soma, entre as mesmas em função do deslocamento entre estas. \noindent Para sequências discretas $x[n]$ e $h[n]$ com $n \in \mathbb{Z}$ a convolução é definida por: $$\displaystyle y[n] = x[n] * h[n] = \sum_{k = -\infty}^{k = \infty} x[k] \cdot h[n -k]$$

No estudo de sistemas lineares e invariantes no tempo a convolução descreve a saída de um dado sistema sendo a sinal de entrada convolvido com a resposta ao impulso do sistema. Seja um sistema causal com sua resposta ao impulso dada por $h[n] = 2\delta[n] + \delta[n-1]$ e entrada $x[n] = 3\delta[n] + 4\delta[n-1] + 5\delta[n-2]$

\begin{figure*}[hb]
  \centering
  \begin{subfigure}[h]{0.3\textwidth}
    \begin{center}
      \begin{pspicture}(0,0)(3,5.75)
        \psaxes{-}(0,0)(3,0)
        \psline[linewidth=1pt,showpoints=true](0,0)(0,3) % x[n]
        \psline[linewidth=1pt,showpoints=true](1,0)(1,4) % x[n-1]
        \psline[linewidth=1pt,showpoints=true](2,0)(2,5) % x[n-2]
        \rput(0,3.5){3} % weight of x[n]
        \rput(1,4.5){4} % weight of x[n-1]
        \rput(2,5.5){5} % weight of x[n-2]
      \end{pspicture}
    \end{center}
    \caption{x[n]}
  \end{subfigure}
  %
  \begin{subfigure}[h]{0.3\textwidth}
    \begin{center}
      \begin{pspicture}(0,0)(3,5.75)
        \psaxes{-}(0,0)(3,0)
        \psline[linewidth=1pt,showpoints=true](0,0)(0,2) % h[n]
        \psline[linewidth=1pt,showpoints=true](1,0)(1,1) % h[n-1]
        \rput(0,2.5){2} % weight of h[n]
        \rput(1,1.5){1} % weight of h[n-1]
      \end{pspicture}
    \end{center}
    \caption{h[n]}
  \end{subfigure}
  \caption{Sinais de entrada x[n] e resposta ao impulso do sistema h[n]}
\end{figure*}

%%% Local Variables:
%%% mode: latex
%%% TeX-master: "../main"
%%% End:

>>>>>>> 5961b38f55ae8b9c7785c30e4a57928189a5e71a
}
%
%\bibliographystyle{ieeetr}
%\bibliography{refpld}
%\addcontentsline{toc}{section}{Referências Bibliográficas}
%
\end{document}

%%% Local Variables:
%%% mode: latex
%%% TeX-master: t
%%% End: 
