\documentclass[a4paper,12pt]{article}
\usepackage[T1]{fontenc}
\usepackage[utf8]{inputenc}
\usepackage[brazil]{babel}
\usepackage{indentfirst}
\usepackage[usenames,dvipsnames,svgnames,table,x11names]{xcolor}
%\usepackage{amsmath,amsfonts,amssymb,enumerate,MnSymbol}
%\usepackage{indentfirst,siunitx}
%\usepackage{pst-sigsys,pst-all,graphicx}
%\usepackage[dvips]{graphicx}
%\usepackage{booktabs,pstricks-add,pst-coil,siunitx,multirow,steinmetz}
%\usepackage[usenames,dvipsnames,svgnames,table,x11names]{xcolor}
%\usepackage[brazil]{varioref}
%\usepackage{pstricks-add,subfig,caption,url,verbatim,textcomp}
%\usepackage[cdot,squaren,Gray]{SIunits}
%\usepackage[framed, numbered, autolinebreaks, useliterate]{mcode}
% Setup pages
\setlength{\hoffset}{0cm}
\setlength{\voffset}{0cm}
\setlength{\textheight}{20.0cm}
\setlength{\textwidth}{14.5cm}
\pagestyle{empty}
% Setup Table Cells
% \newcolumntype{L}[1]{>{\raggedright\let\newline\\\arraybackslash\hspace{0pt}}m{#1}}
% \newcolumntype{C}[1]{>{\centering\let\newline\\\arraybackslash\hspace{0pt}}m{#1}}
% \newcolumntype{R}[1]{>{\raggedleft\let\newline\\\arraybackslash\hspace{0pt}}m{#1}}
%
\hyphenation{}
%
\newtheorem{defin}{Definition}[section]
\newtheorem{teorema}{Theorem}[section]
\renewcommand{\labelitemi}{$\bullet$}
\renewcommand{\labelitemii}{$\cdot$}
\renewcommand{\labelitemiii}{$\diamond$}
\renewcommand{\labelitemiv}{$\ast$}
%
%\setlist[itemize,1]{leftmargin=\dimexpr 26pt-.5in}
%
%\graphicspath{{figures/}}
%\DeclareGraphicsExtensions{.eps}
%%\listfiles
%
\begin{document}
{
\begin{center}{\huge {\color{red}{Escravidão no Brasil}}}\\{\large Trabalho de Ética}\end{center}

\vspace*{1cm}

A escravidão no Brasil, dita oficial ou legalizada, compreende o período colonial, onde os colonizadores europeus utilizavam-se principalmente da mão de obra indígena, até o segundo reinado sendo encerrado com a lei áurea, proclamada em $13$ de maio de $1888$. Durante o período, que compreende entre os séculos XVI e XIX, foram transportados como escravos cerca de três milhões e meio de africanos para o Brasil.%citação%

Para compreender esse período da história do Brasil, se faz necessário observar as questões econômicas e políticas que dominavam o cenário mundial. Na Europa os principais estados começavam um período de expansão de sua influência político-econômica. Os impérios britânico, francês e espanhol disputavam a hegemonia seguidos pelo prósperos comerciantes holandeses e belgas. A coroa portuguesa, que durante o período das grandes navegações, conquistou vários e extensos territórios na África, principalmente na parte ocidental chamada Mina e centro-ocidental chamada Angola de onde os escravos eram embarcados para o Brasil, através da rota atlântica, para as principais cidades portuárias, a saber Recife, Salvador e Rio de Janeiro de onde eram distribuídos para as demais regiões. Vale destacar que o comércio de escravos africanos é muito mais antigo, tendo sendo praticado na África pelos vários reinos, nações e tribos pré-existentes e difundido pelos árabes no qual a força laboriosa era feita por escravos, tendo pois o tráfico na Europa e colônias na América sido introduzido pelos comerciantes árabes. As colônias portuguesas na África e Ásia eram muito pouco atrativas para a ambição comercial da coroa portuguesa. O transporte e comércio de especiarias era a basicamente a única atividade lucrativa nessas colônias. No Brasil a cana-de-açúcar, que provocou invasões holandesas e francesas, em seguida a descoberta de ouro e outros metais preciosos fez com que a coroa portuguesa investisse o necessário para expandir essas lucrativas atividades econômicas.

A força de trabalho africana foi a base para o desenvolvimento do Brasil colônia e imperial. Durante o primeiro período o comércio se concentrou em fornecer a força de trabalho para o nordeste brasileiro, a saber a zona da mata, no período conhecido como ciclo da cana-de-açúcar. O fim desse período coincide com o ciclo do ouro e em seguida a chegada da família real portuguesa ao Brasil e o período cafeeiro provocando a mudança da rota dos escravos para o centro-sul brasileiro. A disponibilidade de escravos africanos, o comércio dos escravos era muito lucrativo para a cora portuguesa, tornou o processo de cativar e traficar índios como força de trabalho muito difícil e caro. Também a resistência dos jesuítas, representantes da igreja católica apostólica romana no brasil, a servidão por outrem e por fim o decreto de 1757 do Marquês de Pombal que proibia o aprisionamento e tráfico de índios. Mesmo assim pode-se identificar que principalmente na região amazônica, províncias do Maranhão e Grão-Pará, uma presença significativa de força de trabalho escravo indígena principalmente na criação do gado e plantações de algodão.

O contexto do fim da escravidão no Brasil se dá em torno da revolução industrial, tendo a grã-bretanha como principal expoente. A busca por desenvolver mercados para os produtos industrializados, fez com que o império britânico vislumbrasse no fim da escravidão uma oportunidade de ampliar a base consumidora desses produtos. Um escravo, por definição, não é um consumidor em potencial mas um trabalhador assalariado o é. Assim em $1845$ o império britânico impõe a lei, conhecida como Bill Aberdeen, que proibia o transporte de escravos entre a África e a América através do oceano atlântico e em $1850$ a lei Eusébio de Queirós proibia a chegada de navios ditos negreiros aos portos brasileiros, sendo seguido por outras leis tais como as do ventre livre e a dos sexagenários culminando com a lei Áurea. Deve-se salientar que durante o período dito abolicionista, houveram vários movimentos político-sociais para continuidade da atividade escravocrata, tendo como principal argumento o direito de posse ou seja um escravo seria um bem, tendo seu valor apropriado, tão quanto a posse de terra era um direito do cidadão brasileiro, outro argumento era de que os africanos não teriam capacidade de adequar a vida de um cidadão comum ou seja eles teriam a necessidade de ter alguém, um descendente de europeu, que controlaria a maneira de como o mesmo deveria se comportar na sociedade.

Com o fim da escravidão temos um período de mudança na sociedade brasileira que tinha um novo personagem o escravo liberto. Nesse período de pouco mais de cem anos a sociedade brasileira tem demonstrado pouca evolução quanto a inclusão social do negro e do índio, citando o presidente Washington Luís \emph{Problema social é um problema de polícia}. Os índices de escolaridade, mortalidade infantil e de desenvolvimento humano quando comparado ao descendente de europeu são significativamente menores bem como as oportunidades de ascensão social são muito limitadas para os descendentes de escravos e índios.
%
%\bibliographystyle{ieeetr}
%\bibliography{refpld}
%\addcontentsline{toc}{section}{Referências Bibliográficas}
%
\end{document}

%%% Local Variables:
%%% mode: latex
%%% TeX-master: t
%%% End: