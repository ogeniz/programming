\begin{enumerate}[1.]

\item Os campos elétricos de duas ondas eletromagnéticas de polarização linear, propagando em unísono através do espaço livre são dados por $E_{1} = e^{-\jmath \pi (z - 0,25)} \hat{a}_{x} [\si{\volt/\meter}]$ e $E_{2} = ae^{-\jmath \pi (z - 0,25b)} \hat{a}_{y} [\si{\volt/\meter}]$ com z em metros e onde a e b são constantes. Determine para a onda eletromagnética resultante: o campo magnético instantâneo, a polarização da onda, o vetor de Poynting instantâneo e médio.  

\item Uma onda eletromagnética plana uniforme harmônica no tempo na frequência \emph{f} propaga na direção $z > 0$ através de um tecido biológico de parâmetros desconhecidos. O campo magnético da onda tem somente a componente $x$ e sua intensidade \emph{rms} na origem do sistema de coordenadas é $H_{0}$ $=$ $25$\si{\milli}[\si{\ampere/\meter}]. A amplitude da onda é reduzida em $3,25$ \si{\decibel} para cada centímetro percorrido, e o coeficiente de fase da onda chega a $\beta$ $=$ $260$[\si{\radian/\meter}]. Determinar a permissividade e a condutividade do tecido.

\item Uma onda eletromagnética plana uniforme harmônica de frequência \emph{f} e intensidade de campo elétrico $E_{i} = 1$\si{\volt/\meter} propaga no ar e incide normalmente na superfície planar de um grande bloco de concreto com parâmetros elétricos $\epsilon_{r} = 6$, $\mu_{r} = 1$ e $\sigma_{r} = 2,5 \times 10^{-3}$. Determinar:

\begin{enumerate}[a.]
{\setlength\itemindent{25pt} \item A classificação do bloco de concreto}
{\setlength\itemindent{25pt} \item A taxa de onda estacionária no ar}
{\setlength\itemindent{25pt} \item O vetor de Poynting médio no tempo no concreto}
{\setlength\itemindent{25pt} \item As percentagens da potência incidente média no tempo que são refletidas a partir da interface e transmitidas no bloco de concreto}
\end{enumerate}

\item Uma onda eletromagnética plana na frequência \emph{f} incide normalmente em um sistema de $3$ meios caracterizados por $\epsilon^{1}_{r}$,  $\epsilon^{2}_{r}$ e $\epsilon^{3}_{r}$, figura $2.40$ da apostila. Projete a camada central (meio $2$) com a menor espessura, para que não haja reflexão de onda incidente de volta ao meio $1$. Gere o gráfico da potência refletida no meio $1$ para a frequência $0,5$ \emph{f} a $1,5$ \emph{f}. Mostre no gráfico a faixa em que a potência refletida é metade da incidente, isto é, a faixa de queda de $3$ \si{\decibel} (caso não encontre aumete a faixa de frequência).

\item Uma onda plana na frequência \emph{f} propaga no ar e incide num ângulo de $\theta_{i}$ sobre o mar ($\epsilon = 80 \epsilon_{0}$, $\sigma = 3$[\si{\siemens/\meter}]), calcular:

\begin{enumerate}[a.]
{\setlength\itemindent{25pt} \item As amplitudes das ondas refletidas e transmitidas ($E_{r}$, $H_{r}$, $E_{t}$, $H_{t}$) considerando que $E_{i} = 500$\si{\micro}[\si{\volt/\meter}] e sua polarização é paralela}
{\setlength\itemindent{25pt} \item O ângulo de transmissão}
\end{enumerate}

\item Uma onda plana na frequência \emph{f} propaga no ar e incide sobre o poliestireno ($\epsilon = 2\epsilon_{0}$) segundo um ângulo de incidência $\theta_{i}$. A onda incidente possui um campo elétrico com $E_{i} = 100$\si{\micro}[\si{\volt/\meter}] e polarizado perpendicularmente, calcular:

\begin{enumerate}[a.]
{\setlength\itemindent{25pt} \item O ângulo de transmissão}
{\setlength\itemindent{25pt} \item As ondas refletidas e transmitidas ($E_{r}$, $H_{r}$, $E_{t}$, $H_{t}$)}
{\setlength\itemindent{25pt} \item O coeficiente de transmissão de energia, isto é, a razão $\dfrac{P^{t}_{m}}{P^{i}_{m}}$}
\end{enumerate}

\end{enumerate}
    

\begin{center}\fbox{\Large RESPOSTAS}\end{center}

\vspace*{0.5cm}

\begin{enumerate}[1.]

\item Dados: $a = 3$, $b = 1$

O campo magnético é determinado por:

\begin{eqnarray*}
  \omega \mu_{0} \vec{H} & = &  \vec{k} \times \vec{E} \nonumber \\
  \dfrac{k \mu_{0}}{\sqrt{\epsilon_{0}\mu_{0}}} \vec{H} & = & k\hat{a}_{z} \times (\hat{a}_{x} + 3\hat{a}_{y})e^{-\jmath (z - 0,25)\pi} \nonumber \\
  \sqrt{\dfrac{\mu_{0}}{\epsilon_{0}}} \vec{H} & = & \left[ (\hat{a}_{z} \times \hat{a}_{x}) + (\hat{a}_{z} \times 3\hat{a}_{y}) \right]e^{-\jmath (z - 0,25)\pi} \nonumber \\
  \vec{H} & = & \sqrt{\dfrac{\epsilon_{0}}{\mu_{0}}}e^{-\jmath (z - 0,25)\pi}\left(-3\hat{a}_{x} + \hat{a}_{y}\right) \nonumber
\end{eqnarray*}

Como $\sqrt{\dfrac{\epsilon_{0}}{\mu_{0}}}$ $=$ $\dfrac{1}{120\pi}$, temos $\vec{H}$ $=$ $\dfrac{1}{40\pi}\left(-\hat{a}_{x} + \dfrac{1}{3}\hat{a}_{y}\right)e^{-\jmath (z - 0,25)\pi} [\si{\ampere/\meter}]$. Com $k_{z} = \pi [\si{\radian/\meter}]$, temos $\omega = 3\pi \times 10^{8} [\si{\radian/\second}]$, assim $\vec{H}$ $=$ $\dfrac{1}{40\pi}\left(-\hat{a}_{x} + \dfrac{1}{3}\hat{a}_{y}\right)\cos[3\pi \times 10^{8}t - (z - 0,25)\pi] [\si{\ampere/\meter}]$

Seja o campo elétrico $\vec{E}$ dado por $\vec{E} = E_{0}x \hat{a}_{x} + E_{0}y \hat{ a_{y}}$  $=$ $e^{-\jmath (z - 0,25) \pi} \hat{a}_{x} + 3 e^{-\jmath (z - 0,25)\pi} \hat{a}_{y} [\si{\volt/\meter}]$, reescrevendo o campo elétrico temos $\vec{E} = (\hat{a}_{x} + A e^{\jmath \theta} \hat{a}_{y})e^{-\jmath k_{z} z}$, com $A = \dfrac{E_{0}y}{E_{0}x}$ $=$ $\dfrac{3 e^{-\jmath (z - 0,25)\pi}}{e^{-\jmath (z - 0,25) \pi}} = 3$, e $\theta = \phi_{y} - \phi{x}  = 0$ e $k_{z} = \pi$, logo $\vec{E} = (\hat{a}_{x} + 3 \hat{a}_{y})e^{-\jmath \pi z}$. A polarização da onda é linear, uma reta que passa pela origem e faz um ângulo $tg^{-1}(3) \approx 71.56\si{\degree}$ com o eixo horizontal.

O vetor de Poynting instantâneo é dado por $\vec{P}(\vec{r},t)$ $=$ $\vec{E}(\vec{r},t) \times \vec{H}(\vec{r},t)$. No problema temos $\vec{E}(\vec{r},t)$ $=$ $\left(\hat{a}_{x} + 3\hat{a}_{y}\right)\cos[3\pi \times 10^{8}t - (z - 0,25)\pi] [\si{\volt/\meter}]$ e $\vec{H}(\vec{r},t)$ $=$ $\dfrac{1}{40\pi}\left(-\hat{a}_{x} + \dfrac{1}{3}\hat{a}_{y}\right)\cos[3\pi \times 10^{8}t - (z - 0,25)\pi] [\si{\ampere/\meter}]$, logo:

\begin{eqnarray*}
  \vec{P}(\vec{r},t) & = & \vec{E}(\vec{r},t) \times \vec{H}(\vec{r},t) \nonumber \\
  & = & \dfrac{1}{40\pi}\cos^{2}[3\pi \times 10^{8}t - (z - 0,25)\pi][(\hat{a}_{x} \times \dfrac{1}{3} \hat{a}_{y}) + (-\hat{a}_{x} + 3\hat{a}_{y})] \nonumber \\
  & = & \dfrac{1}{40\pi}\cos^{2}[3\pi \times 10^{8}t - (z - 0,25)\pi](\dfrac{1}{3}\hat{a}_{z} + 3\hat{a}_{z}) \nonumber \\
  \vec{P}(\vec{r},t) & = &  \dfrac{1}{12\pi}\cos^{2}[3\pi \times 10^{8}t - (z - 0,25)\pi] \hat{a}_{z} [\si{\watt/\meter^{2}}] \nonumber
\end{eqnarray*}

Para o cálculo da potência média fazemos uso da notação fasorial, tal que ${\bf P}_{m} = \dfrac{1}{2}\mathfrak{Re}\left({\bf E} \times {\bf H}^{*} \right)$:

\begin{eqnarray*}  
  {\bf P}_{m} & = & \dfrac{1}{2}\left[e^{-\jmath (z - 0,25)\pi}(\hat{a}_{x} + 3\hat{a}_{y}) \times \dfrac{e^{\jmath (z - 0,25)\pi}}{40\pi}\left(-\hat{a}_{x} + \dfrac{1}{3}\hat{a}_{y}\right)\right] \nonumber \\
  & = & \dfrac{e^{-\jmath (z - 0,25)\pi} e^{\jmath (z - 0,25)\pi}}{80\pi} [(\hat{a}_{x} \times \dfrac{1}{3} \hat{a}_{y}) + \left(-\hat{a}_{x} + 3\hat{a}_{y}\right)] \nonumber \\
  & = & \dfrac{1}{80\pi}\left(\dfrac{1}{3}\hat{a}_{z} + 3\hat{a}_{z}\right) \nonumber \\
  & = & \dfrac{1}{24\pi} \hat{a}_{z}[\si{\watt/\meter^{2}}] \nonumber
\end{eqnarray*}

\item Dados: $f = 1,6$ \si{\giga\hertz}

\newpage

\item Dados: $f = 5$ \si{\giga\hertz}, meio1 caracterizado por $\epsilon_{0}$, $\mu_{0}$ e $\sigma = 0$ e o meio2 por $6 \epsilon_{0}$, $\mu_{0}$, $\sigma = 2,5 \times 10^{-3}$ [\si{\siemens/\meter}]

Para classificar o bloco de concreto, usamos o seguinte critério:
\begin{itemize}
  {\setlength\itemindent{25pt}}\item Se $\dfrac{\sigma}{\omega \epsilon}$ $<$ $0.01$, o material é um dielétrico
  {\setlength\itemindent{25pt}}\item Se $0.01$ $<$ $\dfrac{\sigma}{\omega \epsilon}$ $<$ $100$, o material é quase condutor
  {\setlength\itemindent{25pt}}\item Se $\dfrac{\sigma}{\omega \epsilon}$ $>$ $100$, o material é condutor
\end{itemize}

para o bloco de concreto temos:

\begin{eqnarray*}
  \dfrac{\sigma}{\omega \epsilon} & = & \dfrac{2,5 \times 10^{-3}}{\dfrac{\pi \times 10^{10} \times 10^{-9}}{6\pi}} \nonumber \\
  \dfrac{\sigma}{\omega \epsilon} & \approx & 0,0015 \nonumber
\end{eqnarray*}

como $0.0015$ $<$ $0.01$, logo o bloco de concreto para as condições previamente dadas é um dielétrico.

Para o cálculo da taxa de onda estacionária no ar seguimos a seguinte relação: TOE $=$ $\dfrac{|E|_{M}}{|E|_{m}}$ $=$$\dfrac{|E|^{+}_{0} + |E|^{-}_{0}}{|E|^{+}_{0} - |E|^{-}_{0}}$, com o coeficiente de reflexão definido como $|\Gamma|$ $=$ $\left|\dfrac{E^{-}_{0}}{E^{+}_{0}}\right|$, substituindo temos: TOE $=$ $\dfrac{1 + |\Gamma|}{1 - |\Gamma|}$. Em termos das impedâncias dos meios separados pela interface temos que o $\Gamma$ $=$ $\dfrac{Z_{2} - Z_{1}}{Z_{2} + Z_{1}}$, como $Z$ $=$ $\sqrt{\dfrac{\mu_{0}}{\epsilon_{0}}}$$\sqrt{\dfrac{\mu_{r}}{\epsilon_{r}}}$, assim $Z$ $=$ $120\pi \sqrt{\dfrac{\mu_{r}}{\epsilon_{r}}}$.  Como o meio 1 é o ar, logo $Z_{1}$ $=$ $120\pi$ [\si{\ohm}] e $Z_{2}$ $=$ $\dfrac{120\pi}{\sqrt{6}}$ [\si{\ohm}], assim:

\begin{eqnarray*}
  \Gamma & = & \dfrac{\dfrac{120\pi}{\sqrt{6}} - 120\pi}{\dfrac{120\pi}{\sqrt{6}} + 120\pi} \nonumber \\
  \Gamma & = &\dfrac{\dfrac{1}{\sqrt{6}} - 1}{\dfrac{1}{\sqrt{6}} + 1} \nonumber \\
  \Gamma & \approx & -0,42 \nonumber
\end{eqnarray*}

Substituindo, temos:

\begin{eqnarray*}
  TOE & = & \dfrac{1 + |\Gamma|}{1 - |\Gamma|} \nonumber \\
  TOE & = & \dfrac{1 + 0.42}{1 - 0.42} \nonumber \\
  TOE & \approx & 2,44
\end{eqnarray*}

\end{enumerate}

%%% Local Variables: 
%%% mode: latex
%%% TeX-master: "../main"
%%% End: 
